\subsection{Limited and Homogeneous Participant Pools}
Another recurring limitation is the use of small and often homogeneous participant groups. Many studies employ relatively small sample sizes, sometimes with fewer than 20 or even 15 subjects \citep{Bajaj_S_2018,
%	 Bajaj_S_2022, Chen_J_2018_2, Chen_J_2018_3, Chen_J_2021, Chen_J_2022_2, Chen_J_2023, Gao_Y_2020, Gao_Z_2019_3, Hu_J_2017, Kim_D_2022, Kouchaki_S_2018, Lachtar_R_2022, Latreche_I_2022, Li_Y_2023_1, Liang_Y_2019, Ma_Y_2019, Ma_Y_2020, Mu_Z_2017, Mu_Z_2017_1, Nguyen_H_2023_1, Nguyen_T_2017, Qin_X_2023, Ren_Z_2021, Sanchez_E_2019, Sheykhivand_S_2022, Sheykhivand_S_2022_2, Wang_F_2021, Zou_S_2020, Zhuang_Z_2024, Zhao_C_2009, Zhang_M_2022, Yogarajan_G_2024
}. Such limited sample sizes raise concerns about the statistical power and generalizability of the findings to wider populations. Furthermore, participant pools are often demographically restricted, frequently consisting of young, healthy male university students \citep{Chen_J_2018_2, Chen_J_2018_3, Gao_Y_2020, Gao_Z_2019_3, Hu_J_2017, Igasaki_I_2019, Kim_D_2022, Routray_A_2020, Wang_F_2020, Wang_P_2018, Zuraida_R_2019, Afiuddin_2023, Latreche_I_2025, Giannakopoulou_O_2024, Zou_S_2020}. This homogeneity limits the applicability of the results to female drivers, older drivers, or individuals with varying health conditions or driving experience, all of whom constitute significant portions of the driving population \citep{Arif_S_2023, Eldeib_A_2016, Bajaj_S_2022, Chai_R_2017_1, Kim_D_2022, Luo_H_2019, Wang_F_2020, Zuraida_R_2019, Afiuddin_2023, Latreche_I_2025, Giannakopoulou_O_2024}. The narrow age range and specific demographics might not capture the full spectrum of fatigue manifestation across diverse populations.

\subsection{Subjectivity in Fatigue Assessment and Labeling}
The accurate labeling of fatigue states presents another challenge. Many studies rely on subjective measures like the Karolinska Sleepiness Scale (KSS) \citep{Fu_R_2016, Igasaki_I_2019} or Observer Rating of Drowsiness (ORD) \citep{Houshmand_S_2021, Rezaee_Q_2022, Houshmand_S_2022} for ground truth labeling. While these are established methods, their inherent subjectivity can introduce variability and potential inaccuracies in fatigue state classification \citep{Cai_J_2020, Nugraha_B_2016, Gwak_J_2020}.  Furthermore, relying solely on reaction time (RT) as a proxy for mental state might not perfectly reflect the complex nature of fatigue \citep{Cui_J_2022, Cui_J_2023_1}, and defining fatigue states based on time-on-task or subjective questionnaires may not fully capture the physiological fatigue level \citep{Wang_P_2018, Zhao_C_2011}.  The inconsistencies between subjective fatigue reports and actual fatigue levels, potentially due to emotional and psychological states, further complicate accurate labeling and model training \citep{Min_J_2021, Nugraha_B_2016}.

\section{Methodological and Technical Constraints}

\subsection{Computational Complexity and Real-time Applicability}
The computational demands of some advanced methods pose a significant hurdle for real-time driver fatigue detection systems.  Deep learning models, while often achieving high accuracy, can be computationally intensive, raising concerns about their feasibility for deployment in resource-constrained in-vehicle systems \citep{Balam_V_2021, Budak_U_2019, Gao_D_2023, Jia_H_2023, Ko_W_2020, Xie_H_2024, Ye_H_2024, Zhang_W_2020, Alghanim_M_2024, Chen_K_2024, Gong_P_2024, He_L_2024}.  Specifically, methods like Capsule Networks \citep{Pan_J_2023}, Graph Convolutional Networks (GCNs) \citep{Chen_K_2024, Xie_H_2024}, Transformer-based models \citep{Wang_J_2022, Zhang_J_2025}, and complex feature selection algorithms \citep{Tuncer_T_2021_1, Zeng_H_2021} can introduce substantial computational overhead.  The complexity of feature extraction techniques, such as CEEMDAN \citep{Liu_Y_2024}, TQWT \citep{Bajaj_V_2020, Budak_U_2019}, and dynamic brain network construction \citep{Lin_Z_2021}, also contributes to the computational burden.  Batch normalization, while beneficial for training, can hinder online applications requiring real-time processing \citep{Cui_J_2022}.  Furthermore, the optimization of hyperparameters and network architectures can be time-consuming and computationally expensive \citep{Li_R_2023_1, Yang_Y_2021, Zhao_Y_2023, Alghanim_M_2024}, and the need for efficient data augmentation strategies adds to the computational considerations \citep{Alghanim_M_2024, Seo_P_2024}.  The lack of real-time validation for many proposed methods further underscores the gap between offline analysis and practical implementation \citep{Bajaj_V_2020, Bajaj_S_2022, Cui_J_2022, Darojatun_A_2023, Qin_Y_2022, Vinod_K_2022, Vinod_S_2022, Ma_Y_2019, Nguyen_T_2017}.

\subsection{Signal Modality and Data Quality}
Many studies rely solely on single-channel EEG data, often from frontal electrodes, for drowsiness detection \citep{Bajaj_S_2022, Chinara_V_2021_2, Cui_J_2021, Ding_S_2019, Elidrissi_M_2023, Mishra_C_2024, Mitsukura_M_2018, Yang_H_2024_1}. While single-channel EEG systems offer practical advantages in terms of wearability and user comfort, they might inherently capture less comprehensive brain activity compared to multi-channel EEG setups, potentially limiting the detection of complex fatigue-related patterns \citep{Bajaj_S_2022, Ding_S_2019, Mishra_C_2024, Mitsukura_M_2018, Yang_H_2024_1, Cui_J_2021}.  Furthermore, focusing primarily on central electrode channels or frontal regions may overlook relevant fatigue-related changes in other brain areas \citep{Rahman_F_2024, Tang_J_2024, Zuraida_R_2019}.  The quality of EEG data itself presents a significant challenge. EEG signals are inherently feeble and susceptible to various sources of noise and artifacts, including sensor noise, EMG activity, motion artifacts, and electromagnetic interference \citep{Cui_J_2023_1, Majumder_S_2019, Wang_F_2018, Wang_F_2023, Wang_F_2023_1, Ghadami_A_2022, Feng_X_2025, Wang_F_2018}. Even with preprocessing techniques, the non-stationary nature of EEG signals and inter-subject variability can further degrade detection performance \citep{Chen_C_2023, Dimitrakopoulos_G_2018, Dong_N_2019, Ghadami_A_2022, Paulo_J_2021, Zhuang_Z_2024, Liu_Y_2016, Majumder_S_2019}.  The use of wet electrodes, while providing better signal quality, can be less practical for long-term real-world applications compared to dry electrodes, and issues like scalp attachments and electrode drifts can further compromise signal integrity \citep{Kartsch_V_2018, Peivandi_M_2023, Sheykhivand_S_2022_1, Wang_F_2018, Shahbakhti_M_2022}.  The interpretability of models can also be hindered when applied to noisy or artifact-laden EEG data \citep{Cui_J_2023_1, Zhang_M_2022, Zhuang_Z_2024}.  Moreover, relying solely on EEG signals might overlook valuable information available in other physiological signals, such as EOG, ECG, and EMG, which could enhance the robustness and accuracy of fatigue detection systems \citep{Tang_J_2024, Wang_F_2024_3, Wang_F_2022_1}.

\subsection{Algorithm and Model Limitations}
While sophisticated algorithms and models are being developed, limitations persist in their design and application.  Many studies focus on binary classification (alert vs. fatigued), simplifying the continuous spectrum of fatigue into discrete states and potentially overlooking intermediate fatigue levels \citep{Budak_U_2019, Chai_R_2017_1, Chen_J_2021, Chen_J_2022_2, Chen_J_2023, Cheng_E_2019, Hu_J_2017, Kim_D_2022, Kouchaki_S_2018, Li_R_2023_3, Nguyen_H_2023, Zhang_T_2021_1, Zou_S_2020_1, Chen_L_2015, Chen_D_2025, Zhang_Y_2023_4}.  This binary approach might not fully capture the nuances of fatigue progression and could be less informative for real-world applications requiring graded fatigue assessment \citep{Budak_U_2019, Chai_R_2017_1, Chen_J_2021, Chen_J_2022_2, Chen_J_2023, Cheng_E_2019, Hu_J_2017, Kim_D_2022, Kouchaki_S_2018, Li_R_2023_3, Nguyen_H_2023, Zhang_T_2021_1, Zou_S_2020_1, Chen_L_2015, Chen_D_2025, Zhang_Y_2023_4}.  Linear classifiers like Logistic Regression \citep{Liang_Y_2019} and simpler models like GMLVQ \citep{Golz_M_2020} have shown limitations compared to more complex methods.  Hand-crafted features, while interpretable, may be less robust and generalizable than features learned automatically by deep learning models in certain scenarios \citep{Chinara_V_2021}.  The interpretability of complex deep learning models remains a challenge, making it difficult to understand the specific EEG patterns indicative of fatigue and hindering trust in these systems \citep{Cui_J_2021, Cui_J_2023_1, Ko_W_2020, Sangeetha_S_2023, Sheykhivand_S_2022, Zhang_M_2022}.  Furthermore, the performance of models can be sensitive to hyperparameter tuning and parameter initialization \citep{Bencsik_B_2023, Chen_K_2023, Ma_C_2024, Sharma_S_2021, Song_K_2021, Yuan_L_2024, Feng_X_2025, Gao_D_2024_2, Zhao_Y_2023}, and the selection of optimal data windows for training and analysis requires careful consideration \citep{Jung_S_2017, Liang_Y_2019, Shahbakhti_M_2023, Wang_J_2022}.  The directionality of connections between EEG channels is often not considered \citep{Chen_K_2024}, and simplified channel cooperation mechanisms may limit performance \citep{Pan_Y_2020, Paulo_J_2021}.  The potential for error propagation in self-training methods, especially with inaccurate pseudo-labels, also needs to be addressed \citep{Liu_Y_2024, Feng_X_2025}.

\subsection{Performance Evaluation and Metrics}
Many studies primarily rely on classification accuracy as the primary performance metric \citep{Lins_I_2024, Yang_K_2024_1, Gao_Z_2020_2, Min_J_2018, Rashid_M_2021, Rahmatillah_O_2019, Foong_R_2019, Gu_T_2024, Hussein_R_2024}. While accuracy provides a general indication of performance, it may not be sufficient for a comprehensive evaluation, especially in imbalanced datasets where fatigue events might be less frequent than alert states \citep{Foong_R_2019, Liang_Y_2019, Gong_P_2024}.  The lack of reporting of other crucial metrics such as precision, recall, F1-score, specificity, sensitivity, and AUC \citep{Fu_R_2016, Gao_Z_2020_2, Min_J_2018, Ojha_D_2023, Rashid_M_2021, Rahmatillah_O_2019, Wang_F_2014, Luo_H_2019, Yang_H_2024, Gu_T_2024, Hussein_R_2024, Ramos_P_2022} limits a deeper understanding of the model's performance, particularly in terms of false positives and false negatives, which have different implications in safety-critical applications like driver fatigue detection.  Furthermore, performance measures are often dependent on specific trial and dataset partitioning, making direct comparisons across studies challenging \citep{Hong_S_2018, Mehreen_A_2019, Subasi_A_2022}.  The absence of standardized datasets and evaluation protocols further hinders the comparability and benchmarking of different methods \citep{Fu_R_2016, Majumder_S_2019, Mehreen_A_2019, Subasi_A_2022, Rahmatillah_O_2019, Ojha_D_2023}.

\section{Multimodal and System-Level Challenges}

\subsection{Integration of Multimodal Data and Wearability}
While multimodal approaches, combining EEG with other modalities like eye-tracking, ECG, EOG, and vehicle dynamics, hold promise for improved fatigue detection, they also introduce new limitations.  The need for multiple sensors can increase system complexity, cost, and intrusiveness, potentially reducing practicality and user acceptance \citep{Lian_Z_2024, Martensson_H_2019, Monteiro_T_2019, Yin_J_2017, Chan_K_2021, Flumeri_G_2022, Martensson_H_2019}.  The design of systems requiring separate devices for data acquisition, such as EEG caps and eye-tracking glasses, can limit practicality in real-world settings \citep{Lian_Z_2024}.  The temporal alignment and fusion of data from different modalities with varying time windows and sampling rates also pose challenges, potentially adding complexity to online applications \citep{Lian_Z_2024, Tang_J_2021}.  The wearability and comfort of EEG devices, particularly those using wet electrodes, remain concerns for long-term monitoring in real-world driving scenarios \citep{Kartsch_V_2018, Martensson_H_2019, Nguyen_H_2023_1, Shalash_W_2019, Sheykhivand_S_2022_1, Yin_J_2017}.  The development of non-obtrusive and user-friendly wearable EEG systems is crucial for the widespread adoption of driver fatigue detection technology \citep{Chan_K_2021, Shalash_W_2019, Wei_C_2018}.

\subsection{System Robustness and Contextual Factors}
The robustness of driver fatigue detection systems in diverse real-world driving conditions and across varying contexts remains a critical area for improvement.  Factors such as environmental noise, motion artifacts, varying lighting conditions, and diverse driving scenarios can significantly impact system performance \citep{Mehreen_A_2019, Monteiro_T_2019, Nguyen_H_2023, Wang_C_2020_2, Wang_F_2018, Zong_S_2024, Ardabili_S_2024, Ojha_D_2023}.  Inter-subject variability in EEG signals and fatigue manifestation further challenges the development of universally applicable models \citep{Chen_C_2023, Dimitrakopoulos_G_2018, Dong_N_2019, Ghadami_A_2022, Majumder_S_2019, Monteiro_T_2019, Paulo_J_2021, Zhuang_Z_2024, Liu_Y_2016, Gao_Z_2019_2, Hasan_M_2022, Paulo_J_2021}.  The influence of individual physical and rest conditions, emotions, and psychological states on EEG signals and fatigue patterns adds to the complexity of robust fatigue detection \citep{Dimitrakopoulos_G_2018, Dong_N_2019, Min_J_2021, Nugraha_B_2016, Zhang_H_2023_1, Li_P_2023, Zong_S_2024}.  Contextual factors like road type, traffic density, time of day, and driver workload can also modulate fatigue levels and EEG patterns, requiring context-aware and adaptive detection systems \citep{Martensson_H_2019, Mehreen_A_2019, Mu_Q_2021, Zong_S_2024, Guo_Z_2018, Lin_C_2021}.  The need for personalized and subject-adaptive models to address inter-subject variability and individual differences is frequently emphasized \citep{Paulo_J_2021, Wang_F_2020, Wu_D_2017, Feng_W_2024, Gao_D_2024_3, Reddy_T_2022, Zhang_Y_2022_1, Zhao_Y_2021, Shen_M_2021, Zhuang_Z_2024, Li_R_2023_2, Wang_F_2020}.  Furthermore, the long-term wearability, user acceptance, and ethical considerations related to continuous driver monitoring require careful attention for the successful translation of research findings into practical applications \citep{Martensson_H_2019, Nguyen_H_2023, Yin_J_2017, Chung_G_2018, Chung_G_2018_1, Guo_Z_2018, Yin_J_2017}.


